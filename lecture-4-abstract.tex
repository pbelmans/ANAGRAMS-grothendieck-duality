\begin{abstract}
  These are the notes for my fourth and final lecture on Grothendieck duality in the ANAGRAMS seminar. They are dedicated to an overview of some of the proofs in the literature. They are significantly more detailed than the expos\'e in the seminar. I describe:
  \begin{enumerate}
    \item Hartshorne's geometric proof,
    \item Deligne's pro-objects categorical proof,
    \item Neeman's proof based on Brown's representability,
    \item Murfet's proof based on the mock homotopy category of projectives;
  \end{enumerate}
  while the approaches by Lipman and Yekutieli--Zhang are merely mentioned. The goal of discussing these approaches is to highlight the two main contrasting aspects of Grothendieck duality:
  \begin{enumerate}
    \item local versus global;
    \item categorical versus geometric.
  \end{enumerate}
  The interplay of these makes the study of Grothendieck duality so interesting.
\end{abstract}

