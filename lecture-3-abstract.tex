\begin{abstract}
  These are the notes for my third lecture on Grothendieck duality in the ANAGRAMS seminar. We (finally) come to a statement of Grothendieck duality. In order to do so we first review derived categories, from the viewpoint of someone who has already touched homological algebra in the usual sense \cite{weibel-homological-algebra}.

  After this quick reminder some motivation for considering a possible generalisation of Serre duality is discussed, after which the full statement of Grothendieck duality (in various incarnations) is given. To conclude some applications of Grothendieck duality are discussed, from my point of view on the subject. I hope these serve both as a motivation for Grothendieck duality and as a motivation to study these interesting subjects, regardless from their relationship with Grothendieck duality.
\end{abstract}

