\documentclass[10pt,a4paper]{article}
\usepackage{hyperref}
\usepackage{cleveref}
\hypersetup{hypertexnames = false, bookmarksdepth = 2, bookmarksopen = true, colorlinks, linkcolor = black, citecolor = black, urlcolor = black, pdfstartview={XYZ null null 1}}

\usepackage{amsfonts}
\usepackage{amsmath}
\usepackage{amsthm}
\usepackage{biblatex}
\usepackage{enumitem}
\usepackage{mathtools}
\usepackage{thmtools}
\usepackage{tikz-cd}
\usepackage[colorinlistoftodos]{todonotes}
\usepackage{xparse}
\usepackage{xspace}

\usepackage[T1]{fontenc}
\usepackage[charter]{mathdesign}
\usepackage[scaled]{beramono,berasans}
\usepackage{eucal}
\usepackage{epstopdf}
\usepackage{microtype}
\frenchspacing

\addbibresource{bibliography.bib}

\addtolength\parskip{.4ex}
\setlength\parindent{0cm}

\relpenalty=10000
\binoppenalty=10000

% todonotes configuration
\newcounter{todocounter}
\DeclareDocumentCommand\addreference{g}{\stepcounter{todocounter}\todo[color = blue!30, fancyline]{\thetodocounter. Add reference\IfNoValueF{#1}{: #1}}\xspace}
\DeclareDocumentCommand\checkthis{g}{\stepcounter{todocounter}\todo[color = red!50, fancyline]{\thetodocounter. Check this\IfNoValueF{#1}{: #1}}\xspace}
\DeclareDocumentCommand\fixthis{g}{\stepcounter{todocounter}\todo[color = orange!50, fancyline]{\thetodocounter. Fix this\IfNoValueF{#1}{: #1}}\xspace}
\DeclareDocumentCommand\expand{g}{\stepcounter{todocounter}\todo[color = green!50, fancyline]{\thetodocounter. Expand\IfNoValueF{#1}{: #1}}\xspace}
\newcommand\removethis{\stepcounter{todocounter}\todo[color=yellow!50]{\thetodocounter. Remove this?}}

\declaretheoremstyle[
  spaceabove = 3pt,
  spacebelow = 3pt,
]{lecture}
\theoremstyle{lecture}
\newtheorem{theorem}{Theorem}
\newtheorem{corollary}[theorem]{Corollary}
\newtheorem{definition}[theorem]{Definition}
\newtheorem{example}[theorem]{Example}
\newtheorem{lemma}[theorem]{Lemma}
\newtheorem{proposition}[theorem]{Proposition}
\newtheorem{remark}[theorem]{Remark}

\mathchardef\mhyphen="2D
\newcommand\dash{\nobreakdash-\hspace{0pt}}

\DeclareMathOperator\hh{h}
\DeclareMathOperator\HH{H}
\DeclareMathOperator\Spec{Spec}

\title{Grothendieck duality: lecture 1 \\[.2em] \Large Riemann--Roch theorem and Serre duality}
\author{Pieter Belmans}

\begin{document}
\maketitle

\begin{abstract}
  
\end{abstract}

\tableofcontents

\section{Riemann--Roch}
\label{section:riemann-roch}
\subsection{History}
\label{section:riemann-roch-history}

\subsection{Statement: curves}
\label{subsection:statement-curves}
From now on I will use Ravi Vakil's notes on Riemann--Roch and Serre duality \cite{vakil-proof-riemann-roch}. These are by now probably subsumed in his excellent notes \cite{vakil-math216}.

We first need to figure out what~$\HH^0$ and~$\HH^1$ are, in as concrete terms as possible. From now on we take~$C$ a nonsingular projective algebraic curve over an algebraically closed field~$k$.
\paragraph{Global sections}
\begin{definition}
  Let~$\mathcal{F}$ be a sheaf on~$C$. Then~$\HH^0(C,\mathcal{F})$ are the \emph{global sections} of~$\mathcal{F}$ over~$C$.
\end{definition}
\begin{example}
  Let~$C=\mathbb{P}_k^1$. Then the global sections of~$\mathcal{O}_{\mathbb{P}_k^1}$ on~$\mathbb{P}_k^1$ are the constant functions, i.e.
  \begin{equation}
    \HH^0(\mathbb{P}_k^1,\mathcal{O}_{\mathcal{P}_k^1})=k.
  \end{equation}
  To see this, observe that~$\mathbb{P}_k^1$ is a gluing of two~$\mathbb{A}_k^1$'s. Regular functions on~$\mathbb{A}_k^1=\Spec k[x]$ are polynomials. But if we take a polynomial~$f(x)$, the gluing procedure tells us that~$f(1/x)$ should be a polynomial on the other~$\mathbb{A}_k^1$, which is only possible if it is a constant.
\end{example}
We observe that the global sections have the structure of a~$k$\dash vectorspace. This is the case for all~$\mathcal{O}_C$\dash modules. In this case we define
\begin{equation}
  \hh^0(C,\mathcal{F})\coloneqq\dim_k\HH^0(C,\mathcal{F}).
\end{equation}
\begin{example}
  Let~$C$ be an elliptic curve\expand.
\end{example}

\paragraph{$\HH^1$ of a sheaf}
\begin{definition}
  Let~$\mathcal{F}$ be an~$\mathcal{O}_C$\dash module. Let~$\mathfrak{U}=\{U_1,\dotsc,U_n\}$ be an open cover of~$C$. Denote~$U_{i,j}=U_i\cap U_j$ and~$U_{i,j,k}=U_i\cap U_j\cap U_k$. Then~$\HH^1(C,\mathcal{F})$ \emph{as a set} consists of those tuples~$(f_{i,j})_{i,j}$ where~$f_{i,j}\in\HH^0(U_{i,j},\mathcal{F})$ such that~$f_{i,j}-f_{j,k}+f_{i,k}=0$ in~$\HH^0(U_{i,j,k},\mathcal{F})$. We will call these \emph{cocycles}.

  We consider~$\HH^1(C,\mathcal{F})$ \emph{as an abelian group} by declaring a tuple~$(f_{i,j})_{i,j}$ zero if there are sections~$g_i\in\HH^0(U_i,\mathcal{F})$ such that~$f_{i,j}=g_i-g_j$ in~$\HH^0(U_{i,j},\mathcal{F})$. And we get~$\HH^1(C,\mathcal{F})$ \emph{as a~$k$\dash vectorspace} by the~$k$\dash vectorspace structure on~$\HH^0(C,\mathcal{F})$.
\end{definition}
The definition of the zero in this vectorspace explains what~$\HH^1$ is about: it measures to which extent we cannot glue global sections.
% finer coverings
% Leray result

\begin{theorem}[Riemann--Roch]
  \label{theorem:riemann-roch}
  Let~$\mathcal{L}$ be an invertible sheaf of degree~$d$. Then\fixthis{nice definition of $\Omega^1$}
  \begin{equation}
    \hh^0(C,\mathcal{L})-\hh^0(C,\Omega_C^1\otimes\mathcal{L}^\vee)=d-g+1.
  \end{equation}
\end{theorem}
Hence Riemann--Roch is a relationship between two numbers: the easy~$\hh^0(C,\mathcal{L})$ and the difficult~$\hh^1(C,\Omega_C^1\otimes\mathcal{L}^\vee)$.


\subsection{Statement: surfaces}
\label{subsection:statement-surfaces}


\section{Serre duality}
\label{section:serre-duality}
\subsection{History}
\label{subsection-serre-duality-history}

\subsection{Statement}
\label{subsection:serre-duality-statement}

\subsection{Proof of the curve case}
\label{subsection:serre-duality-curves}
The curve case of Serre duality is the following.
\begin{theorem}
  \label{theorem:serre-duality-curves}
  There is a natural perfect pairing
  \begin{equation}
    \HH^0(C,\Omega_C^1(-D))\times\HH^1(C,\mathcal{O}_C(D))\to k.
  \end{equation}
\end{theorem}

\begin{proof}[Proof of Riemann--Roch using Serre duality]
  We have that\fixthis{motivation for some steps}
  \begin{equation}
    \begin{aligned}
      &\hh^0(C,\mathcal{L})-\hh^0(C,\Omega^1\otimes\mathcal{L}^\vee) \\
      &\quad=\hh^0(C,\mathcal{L})-\hh^1(C,\mathcal{L}) & \text{Riemann--Roch} \\
      &\quad=\chi(C,\mathcal{L}) & \text{definition of~$\chi$} \\
      &\quad=d+\chi(C,\mathcal{O}_C) & \text{\dots} \\
      &\quad=d+\hh^0(C,\mathcal{O}_C)-\hh^1(C,\mathcal{O}_C) & \text{definition of~$\chi$} \\
      &\quad=d+1-\hh^1(C,\mathcal{O}_C) & \text{global sections are constants} \\
      &\quad=d+1-\hh^0(C,\Omega_C^1) & \text{Serre duality} \\
      &\quad=d+1-g & \text{definition of $g$}.
    \end{aligned}
  \end{equation}
\end{proof}

The rest of this section is dedicated to the proof of Serre duality in the curve case. It is taken from Vakil's notes, which are based on \cite{serre-groupes-algebriques-et-corps-de-classes} and originate from a proof by Weil.

\subsection{Remarks on the general case}
\label{subsection:remarks}


\section{Applications}
\label{section:applications}

\subsection{Facts about curves}
\label{subsection:facts-curves}

\subsection{Serre functors}
\label{subsection:serre-functors}

\printbibliography

\end{document}
