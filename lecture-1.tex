\documentclass[10pt,a4paper]{article}
\usepackage{hyperref}
\usepackage{cleveref}
\hypersetup{hypertexnames = false, bookmarksdepth = 2, bookmarksopen = true, colorlinks, linkcolor = black, citecolor = black, urlcolor = black, pdfstartview={XYZ null null 1}}

\usepackage{amsfonts}
\usepackage{amsmath}
\usepackage{amsthm}
\usepackage{biblatex}
\usepackage{enumitem}
\usepackage{mathtools}
\usepackage{thmtools}
\usepackage{tikz-cd}
\usepackage[colorinlistoftodos]{todonotes}
\usepackage{xparse}
\usepackage{xspace}

\usepackage[T1]{fontenc}
\usepackage[charter]{mathdesign}
\usepackage[scaled]{beramono,berasans}
\usepackage{eucal}
\usepackage{epstopdf}
\usepackage{microtype}
\frenchspacing

\addbibresource{bibliography.bib}

\addtolength\parskip{.4ex}
\setlength\parindent{0cm}

\relpenalty=10000
\binoppenalty=10000

% todonotes configuration
\newcounter{todocounter}
\DeclareDocumentCommand\addreference{g}{\stepcounter{todocounter}\todo[color = blue!30, fancyline]{\thetodocounter. Add reference\IfNoValueF{#1}{: #1}}\xspace}
\DeclareDocumentCommand\checkthis{g}{\stepcounter{todocounter}\todo[color = red!50, fancyline]{\thetodocounter. Check this\IfNoValueF{#1}{: #1}}\xspace}
\DeclareDocumentCommand\fixthis{g}{\stepcounter{todocounter}\todo[color = orange!50, fancyline]{\thetodocounter. Fix this\IfNoValueF{#1}{: #1}}\xspace}
\DeclareDocumentCommand\expand{g}{\stepcounter{todocounter}\todo[color = green!50, fancyline]{\thetodocounter. Expand\IfNoValueF{#1}{: #1}}\xspace}
\newcommand\removethis{\stepcounter{todocounter}\todo[color=yellow!50]{\thetodocounter. Remove this?}}

\declaretheoremstyle[
  spaceabove = 3pt,
  spacebelow = 3pt,
]{lecture}
\theoremstyle{lecture}
\newtheorem{theorem}{Theorem}
\newtheorem{corollary}[theorem]{Corollary}
\newtheorem{definition}[theorem]{Definition}
\newtheorem{example}[theorem]{Example}
\newtheorem{lemma}[theorem]{Lemma}
\newtheorem{proposition}[theorem]{Proposition}
\newtheorem{remark}[theorem]{Remark}

\mathchardef\mhyphen="2D
\newcommand\dash{\nobreakdash-\hspace{0pt}}

\DeclareMathOperator\hh{h}
\DeclareMathOperator\HH{H}
\DeclareMathOperator\res{res}
\DeclareMathOperator\Spec{Spec}
\DeclareMathOperator\supp{supp}

\title{Grothendieck duality: lecture 1 \\[.2em] \Large Riemann--Roch theorem and Serre duality}
\author{Pieter Belmans}

\begin{document}
\maketitle

\begin{abstract}
  
\end{abstract}

\tableofcontents

\section{Riemann--Roch}
\label{section:riemann-roch}
\subsection{History}
\label{subsection:riemann-roch-history}

\subsection{Preliminaries}
\label{subsection:preliminaries}
From now on I will use Ravi Vakil's notes on Riemann--Roch and Serre duality \cite{vakil-proof-riemann-roch}. These are by now probably subsumed in his excellent notes \cite{vakil-math216}.

We first need to figure out what~$\HH^0$ and~$\HH^1$ are, in as concrete terms as possible. From now on we take~$C$ a nonsingular projective algebraic curve over an algebraically closed field~$k$.
\paragraph{Global sections}
\begin{definition}
  Let~$\mathcal{F}$ be a sheaf on~$C$. Then~$\HH^0(C,\mathcal{F})$ are the \emph{global sections} of~$\mathcal{F}$ over~$C$.
\end{definition}
\begin{example}
  Let~$C=\mathbb{P}_k^1$. Then the global sections of~$\mathcal{O}_{\mathbb{P}_k^1}$ on~$\mathbb{P}_k^1$ are the constant functions, i.e.
  \begin{equation}
    \HH^0(\mathbb{P}_k^1,\mathcal{O}_{\mathcal{P}_k^1})=k.
  \end{equation}
  To see this, observe that~$\mathbb{P}_k^1$ is a gluing of two~$\mathbb{A}_k^1$'s. Regular functions on~$\mathbb{A}_k^1=\Spec k[x]$ are polynomials. But if we take a polynomial~$f(x)$, the gluing procedure tells us that~$f(1/x)$ should be a polynomial on the other~$\mathbb{A}_k^1$, which is only possible if it is a constant.
\end{example}
We observe that the global sections have the structure of a~$k$\dash vectorspace. This is the case for all~$\mathcal{O}_C$\dash modules. In this case we define
\begin{equation}
  \hh^0(C,\mathcal{F})\coloneqq\dim_k\HH^0(C,\mathcal{F}).
\end{equation}
\begin{example}
  Let~$C$ be an elliptic curve\expand.
\end{example}

\paragraph{$\HH^1$ of a sheaf}
\begin{definition}
  Let~$\mathcal{F}$ be an~$\mathcal{O}_C$\dash module. Let~$\mathfrak{U}=\{U_1,\dotsc,U_n\}$ be an open cover of~$C$. Denote~$U_{i,j}=U_i\cap U_j$ and~$U_{i,j,k}=U_i\cap U_j\cap U_k$. Then~$\HH^1(C,\mathcal{F})$ \emph{as a set} consists of those tuples~$(f_{i,j})_{i,j}$ where~$f_{i,j}\in\HH^0(U_{i,j},\mathcal{F})$ such that~$f_{i,j}-f_{j,k}+f_{i,k}=0$ in~$\HH^0(U_{i,j,k},\mathcal{F})$. We will call these \emph{cocycles}.

  We consider~$\HH^1(C,\mathcal{F})$ \emph{as an abelian group} by declaring a tuple~$(f_{i,j})_{i,j}$ zero if there are sections~$g_i\in\HH^0(U_i,\mathcal{F})$ such that~$f_{i,j}=g_i-g_j$ in~$\HH^0(U_{i,j},\mathcal{F})$. And we get~$\HH^1(C,\mathcal{F})$ \emph{as a~$k$\dash vectorspace} by the~$k$\dash vectorspace structure on~$\HH^0(C,\mathcal{F})$.
\end{definition}
The definition of the zero in this vectorspace explains what~$\HH^1$ is about: it measures to which extent we cannot glue global sections.
% finer coverings
% Leray result

\subsection{Statement: curves}
\label{subsection:statement-curves}
We can now give a first version of the Riemann--Roch theorem.
\begin{theorem}[Riemann--Roch]
  \label{theorem:riemann-roch}
  Let~$\mathcal{L}$ be an invertible sheaf of degree~$d$. Then\fixthis{nice definition of $\Omega^1$}
  \begin{equation}
    \hh^0(C,\mathcal{L})-\hh^0(C,\Omega_C^1\otimes\mathcal{L}^\vee)=d-g+1.
  \end{equation}
\end{theorem}
Hence Riemann--Roch is a relationship between two numbers: the easy~$\hh^0(C,\mathcal{L})$ and the difficult~$\hh^1(C,\Omega_C^1\otimes\mathcal{L}^\vee)$.


\subsection{Statement: surfaces}
\label{subsection:statement-surfaces}


\section{Serre duality}
\label{section:serre-duality}
\subsection{History}
\label{subsection-serre-duality-history}

\subsection{Statement}
\label{subsection:serre-duality-statement}

\subsection{Proof of the curve case}
\label{subsection:serre-duality-curves}
The curve case of Serre duality is the following.
\begin{theorem}
  \label{theorem:serre-duality-curves}
  There is a natural perfect pairing
  \begin{equation}
    \HH^0(C,\Omega_C^1(-D))\times\HH^1(C,\mathcal{O}_C(D))\to k.
  \end{equation}
\end{theorem}

\begin{proof}[Proof of Riemann--Roch using Serre duality]
  We have that\fixthis{motivation for some steps}
  \begin{equation}
    \begin{aligned}
      &\hh^0(C,\mathcal{L})-\hh^0(C,\Omega^1\otimes\mathcal{L}^\vee) \\
      &\quad=\hh^0(C,\mathcal{L})-\hh^1(C,\mathcal{L}) & \text{Riemann--Roch} \\
      &\quad=\chi(C,\mathcal{L}) & \text{definition of~$\chi$} \\
      &\quad=d+\chi(C,\mathcal{O}_C) & \text{\dots} \\
      &\quad=d+\hh^0(C,\mathcal{O}_C)-\hh^1(C,\mathcal{O}_C) & \text{definition of~$\chi$} \\
      &\quad=d+1-\hh^1(C,\mathcal{O}_C) & \text{global sections are constants} \\
      &\quad=d+1-\hh^0(C,\Omega_C^1) & \text{Serre duality} \\
      &\quad=d+1-g & \text{definition of $g$}.
    \end{aligned}
  \end{equation}
\end{proof}

The rest of this section is dedicated to the proof of Serre duality in the curve case. It is taken from Vakil's notes, which are based on \cite{serre-groupes-algebriques-et-corps-de-classes} and originate from a proof by Weil.

\paragraph{Ad\`eles}
When I was preparing these notes this part scared me, because ``ad\`ele'' is a scary word I knew from people who know something about class field theory\footnote{It is also the street in which the math department of Universit\'e Paris-Sud is located.}. The approach of the proof of Riemann--Roch taken by Vakil, Serre and Weil is by considering ``pre-ad\`eles'' or repartitions. This avoids the technical machinery of class field theory (it would be insane to use it to prove something as down-to-earth as Riemann--Roch) and has a nice interpretation in terms of the geometry. By preparing these notes I finally got myself familiar with ad\`eles, so I hope other people will benefit too from advertising this approach.

Before we start by building things from the ground up, remark that the occurence of techniques from class field theory is not too far-fetched. It deals with fields, there is a bijection between curves and their function fields\addreference, and the ring of ad\`eles of the function field of a curve satisfies a certain duality\addreference. This duality implies Riemann--Roch, and we will develop as much of class field theory as required for the proof. So let's get started.

\paragraph{The part on $\mathrm{I}(D)$}
\begin{definition}
  A \emph{repartition} is an indexed set~$(f_P)_{p\in C}$ with~$f_P\in k(C)$ such that~$f_P\in\mathcal{O}_{C,P}$ for finitely many points~$p$. The set~$R$ of repartitions comes equipped with a ring structure (pointwise addition and multiplication), with~$k(C)$ being a subring of this (if~$f\in k(C)$ we take~$f_P=f$, which is regular at all but finitely many points of~$C$), and~$R$ being a~$k(C)$\dash algebra.
\end{definition}
Hence a repartition is a collection of rational functions, indexed by the points of the curve, such that at most finitely many rational functions have a pole in the point at which they are associated. This ring contains tons of potential information (recall that~$\HH^1$ was about gluing local sections to a global section, and the obstructions in doing so), and we wish to put it to good use.
\begin{definition}
  Let~$D$ be a divisor on~$C$. We set
  \begin{equation}
    R(D)\coloneqq\{(f_P)_{P\in C}\mid v_P(f_P)+v_P(D)\geq 0\},
  \end{equation}
  an additive subgroup of~$R$.
\end{definition}
This is analogous to~$\mathcal{O}_C(D)$\fixthis{we should have a reminder on the definition somewhere}, but taken for each point separately. We can now interpret~$\HH^1$ in terms of these objects.
\begin{proposition}
  \label{proposition:HH1}
  We have
  \begin{equation}
    \HH^1(C,\mathcal{O}_C(D))\cong R/(R(D)+k(C)).
  \end{equation}

  \begin{proof}
    Associated to the field~$k(C)$ we have the constant sheaf~$\underline{k(C)}$ on~$C$. We have a natural injection of~$\mathcal{O}_C(D)$ into this constant sheaf, and we define~$S$ to be the cokernel of this injection, i.e.\ we have the short exact sequence
    \begin{equation}
      0\to\mathcal{O}_C(D)\to\underline{k(C)}\to S\to 0.
    \end{equation}
    Taking global sections we get
    \begin{equation}
      k(C)\to\HH^0(C,S)\to\HH^1(C,\mathcal{O}_C(D))\to\HH^1(C,\underline{k(C)})=0
    \end{equation}
    because constant sheaves don't have higher cohomology groups. Hence we have to prove that
    \begin{equation}
      \HH^0(C,S)=R/R(D).
    \end{equation}
    To prove this, we have to interpret~$S$ as the quotient sheaf of~$\underline{k(C)}$, which we do by looking at its stalks. If~$P$ is again a point of~$C$, we have
    \begin{equation}
      S_P=k(C)/\mathcal{O}_C(D)=\{f\in k(C)\mid v_P(F)\geq -v_P(D)\}.
    \end{equation}
    Hence\expand{a few more words on motivating this}
    \begin{equation}
      R/R(D)=\bigoplus_{P\in C}S_P.
    \end{equation}

    We wish to show that~$S$ equals this same direct sum of skyscraper sheaves, i.e.\ that sections of~$S$ consist of a selection of values of~$S_P$ for all~$P$, almost all of which are zero. To do so, let~$P$ be a point of~$C$, and let~$s\in S(U)$ be a section defined on an open neighbourhood~$U$ of~$P$. Then there exists a (smaller) neighbourhood~$U'$ such that~$s|_{U'\setminus P}=0$, which means that it has a lift to a section~$s'$ of~$\underline{k(C)}$. It suffices to take an open neighbourhood of~$P$, disjoint of~$(\supp D\setminus\{P\})\cup s^{-1}(\infty)$.
  \end{proof}
\end{proposition}

\paragraph{The part on $\mathrm{J}(D)$}

\paragraph{The part on differentials}

\paragraph{The setup for the final proof}
We wish to show that two vectorspaces are dual to eachother, and we will explicitly construct a linear functional for this. To do so, take~$\omega\in M$\fixthis{notation} a meromorphic differential. We define the divisor
\begin{equation}
  (\omega)\coloneqq\sum_{p\in C}v_P(\omega)P
\end{equation}
hence the sheaf~$\Omega^1(-D)$ is the sheaf of differentials such that~$(\omega)\leq D$.

Then we define the pairing\checkthis{notation?}
\begin{equation}
  \langle-,-\rangle\colon M\times R\to k(C):(\omega,r)\mapsto\langle\omega,r\rangle=\sum_{p\in C}\res_p(r_p\omega).
\end{equation}
It has the following properties.
\begin{lemma}
  \label{lemma:pairing-properties}
  The pairing~$\langle-,-\rangle$ satisfies
  \begin{enumerate}
    \item\label{enumerate:pairing-properties-1} $\langle\omega,r\rangle=0$ if~$r\in k(C)$;
    \item\label{enumerate:pairing-properties-2} $\langle\omega,r\rangle=0$ if~$r\in R(D)$ and~$\omega\in\HH^0(C,\Omega^1(-D))$;
    \item\label{enumerate:pairing-properties-3} $\langle f\omega,r\rangle=\langle\omega,fr\rangle$ if~$f\in k(C)$.
  \end{enumerate}

  \begin{proof}
    \begin{enumerate}
      \item This is the residue theorem, as known from complex analysis. It is also valid in this algebraic setting.
      \item The product~$r_p\omega$ cannot have a pole, for any~$p\in C$, because the zeroes must at least cancel the poles by the assumptions on~$r$ and~$\omega$.
      \item Both pairings evaluate to a sum of residues over~$f\omega r$.
    \end{enumerate}
  \end{proof}
\end{lemma}
For each meromorphic differential~$\omega$ in~$\HH^0(C,\Omega^1(-D))$ we have a linear functional~$\theta(\omega)$ on~$R$, and by \cref{enumerate:pairing-properties-1,enumerate:pairing-properties-2} it is also a linear functional on~$R/(R(D)+k(C))$. Hence we get a map
\begin{equation}
  \theta\colon\HH^0(C,\Omega^1(-D))\to J(D)
\end{equation}
as~$\mathrm{J}(D)$ is shorthand for the dual of~$R/(R(D)+k(C))$\checkthis{or is this a result?}. This~$\theta$ is moreover defined as a map~$M\to J$ in general. But we have the following nice property, that relates the more general map to the specific map.
\begin{lemma}
  \label{lemma:TODO}
  Let~$\omega$ be a meromorphic differential such that~$\theta(\omega)\in\mathrm{J}(D)$. Then~$\omega\in\Omega^1(-D)$\checkthis{shouldn't this be global sections?}.

  \begin{proof}
    Assume on the contrary that~$\omega\notin\Omega^1(-D)$. This means that there is a point~$p\in C$ such that~$\omega$ has a pole in~$p$ that is bigger than allowed by~$D$, or symbolically
    \begin{equation}
      v_p(\omega)<v_p(-D).
    \end{equation}
    Then we take a repartition~$r\in R(D)$ by setting
    \begin{equation}
      r_q=
      \begin{cases}
        0 & q\neq p \\
        1/t^{v_p(\omega)+1}.
      \end{cases}
    \end{equation}
    Because
    \begin{equation}
      v_p(r_p\omega)=-1
    \end{equation}
    we get\checkthis{is this last motivation correct?}
    \begin{equation}
      \langle\omega,r\rangle=\sum_{q\in C}\res(r_q\omega)=\res(r_p\omega)\neq 0.
    \end{equation}
    But this means~$\theta(\omega)$ is not zero on~$R(D)$, but this is required by the definition of~$\theta$, hence we obtain a contradiction.
  \end{proof}
\end{lemma}

Recall that we wish to prove that~$\theta$ induces an isomorphism from~$\HH^0(C,\Omega^1(-D))$ to~$\HH^1(C,\mathcal{O}_C(D))^\vee$, and this last object is also denoted~$\mathrm{J}(D)$.
\begin{proof}[Proof of Serre duality]
  To see that~$\theta$ is \emph{injective}, take~$\omega\in\HH^0(C,\Omega^1(-D))$ such that~$\theta(\omega)=0$. Then by \cref{lemma:theta-J(D)} we have that~$\Omega(-\Delta)$ for every divisor~$\Delta$, which implies~$\omega=0$\expand.

  To see that~$\theta$ is \emph{surjective}, observe that by \cref{TODO} we have that~$\theta$ is~$k(C)$\dash linear, from~$M$ to~$J$\fixthis{notation?}. By \cref{lemma:dim(M)=1} we have~$\dim_{k(C)}M=1$, by \cref{proposition:dim(J)<=1} we have~$\dim_{k(C)}J\leq 1$.
\end{proof}

\subsection{Remarks on the general case}
\label{subsection:remarks}


\section{Applications}
\label{section:applications}

\subsection{Facts about curves}
\label{subsection:facts-curves}

\subsection{Serre functors}
\label{subsection:serre-functors}

\printbibliography

\end{document}
