\documentclass[10pt,a4paper]{article}
\usepackage{hyperref}
\usepackage{cleveref}
\hypersetup{hypertexnames = false, bookmarksdepth = 2, bookmarksopen = true, colorlinks, linkcolor = black, citecolor = black, urlcolor = black, pdfstartview={XYZ null null 1}}

\usepackage{amsfonts}
\usepackage{amsmath}
\usepackage{biblatex}
\usepackage{enumitem}
\usepackage{tikz-cd}
\usepackage[colorinlistoftodos]{todonotes}

\usepackage[T1]{fontenc}
\usepackage[charter]{mathdesign}
\usepackage[scaled]{beramono,berasans}
\usepackage{eucal}
\usepackage{epstopdf}
\usepackage{microtype}
\frenchspacing

\addbibresource{bibliography.bib}

\addtolength\parskip{.4ex}
\setlength\parindent{0cm}

\relpenalty=10000
\binoppenalty=10000

\mathchardef\mhyphen="2D
\newcommand\dash{\nobreakdash-\hspace{0pt}}

\title{Grothendieck duality: lecture 2 \\[.2em] \Large Grothendieck duality and applications}
\author{Pieter Belmans}

\begin{document}
\maketitle

\begin{abstract}
  
\end{abstract}

\tableofcontents

\section{Grothendieck duality}
\label{section:grothendieck-duality}
\subsection{First statement: the twisted inverse image functor}
\label{subsection:first-statement}

\subsection{Related statements: dualising complexes}
\label{subsection:related-statements}

\subsection{How to recover classical results?}
\label{subsection:classical-results}


\section{History of the results}
\label{section:history}


\section{Applications}
\label{section:applications}
\subsection{Local duality}
\label{subsection:local-duality}

\subsection{The yoga of six functors}
\label{subsection:six-functors-yoga}

\subsection{Fourier--Mukai transforms}
\label{subsection:fourier-mukai-transforms}




\printbibliography

\end{document}

