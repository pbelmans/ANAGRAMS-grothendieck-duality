\documentclass[10pt,a4paper]{article}
\usepackage{hyperref}
\usepackage{cleveref}
\hypersetup{hypertexnames = false, bookmarksdepth = 2, bookmarksopen = true, colorlinks, linkcolor = black, citecolor = black, urlcolor = black, pdfstartview={XYZ null null 1}}

\usepackage{amsfonts}
\usepackage[fleqn, leqno]{amsmath}
\usepackage{amsthm}
\usepackage{biblatex}
\usepackage{booktabs}
\usepackage{diagbox}
\usepackage{enumitem}
\usepackage{fixltx2e}
\usepackage{mathtools}
\usepackage{thmtools}
\usepackage{tikz-cd}
\usepackage[colorinlistoftodos]{todonotes}
\usepackage{xparse}
\usepackage{xspace}

\usepackage[T1]{fontenc}
\usepackage[charter]{mathdesign}
\usepackage[scaled]{beramono,berasans}
\usepackage{eucal}
\usepackage{epstopdf}
\usepackage{microtype}
\frenchspacing

\addbibresource{bibliography.bib}

\addtolength\parskip{.4ex}
\setlength\parindent{0cm}

\relpenalty=10000
\binoppenalty=10000

% todonotes configuration
\newcounter{todocounter}
\DeclareDocumentCommand\addreference{g}{\stepcounter{todocounter}\todo[color = blue!30, fancyline]{\thetodocounter. Add reference\IfNoValueF{#1}{: #1}}\xspace}
\DeclareDocumentCommand\checkthis{g}{\stepcounter{todocounter}\todo[color = red!50, fancyline]{\thetodocounter. Check this\IfNoValueF{#1}{: #1}}\xspace}
\DeclareDocumentCommand\fixthis{g}{\stepcounter{todocounter}\todo[color = orange!50, fancyline]{\thetodocounter. Fix this\IfNoValueF{#1}{: #1}}\xspace}
\DeclareDocumentCommand\expand{g}{\stepcounter{todocounter}\todo[color = green!50, fancyline]{\thetodocounter. Expand\IfNoValueF{#1}{: #1}}\xspace}
\newcommand\removethis{\stepcounter{todocounter}\todo[color=yellow!50]{\thetodocounter. Remove this?}}

% environments
\declaretheoremstyle[
  spaceabove = 3pt,
  spacebelow = 3pt,
]{lecture}
\theoremstyle{lecture}
\newtheorem{theorem}{Theorem}
\newtheorem{corollary}[theorem]{Corollary}
\newtheorem{definition}[theorem]{Definition}
\newtheorem{example}[theorem]{Example}
\newtheorem{lemma}[theorem]{Lemma}
\newtheorem{proposition}[theorem]{Proposition}
\newtheorem{remark}[theorem]{Remark}


\mathchardef\mhyphen="2D
\newcommand\dash{\nobreakdash-\hspace{0pt}}
\newcommand\bounded{\ensuremath{\mathrm{b}}}
\newcommand\Coh{\ensuremath{\mathrm{Coh}}}
\newcommand\dd{\mathrm{d}}
\newcommand\derived{\ensuremath{\mathbf{D}}}
\newcommand\KKK{\ensuremath{\mathbf{K}}}
\newcommand\Qcoh{\ensuremath{\mathrm{Qcoh}}}
\newcommand\RRR{\ensuremath{\mathbf{R}}}

\DeclareMathOperator\Ch{Ch}
\DeclareMathOperator\Ext{Ext}
\DeclareMathOperator\hh{h}
\DeclareMathOperator\HH{H}
\DeclareMathOperator\Hom{Hom}
\DeclareMathOperator\tr{tr}
\DeclareMathOperator\Proj{Proj}
\DeclareMathOperator\res{res}
\DeclareMathOperator\Spec{Spec}
\DeclareMathOperator\supp{supp}


\title{Grothendieck duality: lecture 4 \\[.2em] \Large Sketches of some of the proofs}
\author{Pieter Belmans}
\date{January 29, 2014}

\begin{document}
\maketitle

\begin{abstract}
  These are the notes for my fourth and final lecture on Grothendieck duality in the ANAGRAMS seminar. They contain some applications of Grothendieck duality, given from my point of view. I hope these serve both as a motivation for Grothendieck duality and as a motivation to study these interesting subjects, regardless from their relationship with Grothendieck duality.

  The remainder of these notes is dedicated to an overview of some of the proofs in the literature. They are significantly more detailed than the expos\'e in the seminar. I describe:
  \begin{enumerate}
    \item Hartshorne's geometric proof,
    \item Deligne's pro-objects categorical proof,
    \item Neeman's proof based on Brown's representability,
    \item Murfet's proof based on the mock homotopy category of projectives;
  \end{enumerate}
  while the approaches by Lipman and Yekutieli--Zhang are merely mentioned.
\end{abstract}

\tableofcontents

\clearpage

\section{Applications of Grothendieck duality}
\label{section:applications-grothendieck-duality}
Due to time constraints, both in preparing these notes and actually lecturing about them, the following list of applications is not as worked out as I want it to be.

\subsection{The yoga of six functors}
\label{subsection:six-functors-yoga}
The notion of Grothendieck duality that we have seen so far is in the following situation:
\begin{enumerate}
  \item (quasi)coherent sheaves;
  \item Zariski topology for schemes;
\end{enumerate}
But one can consider other contexts too. In the study of \'etale cohomology we have:
\begin{enumerate}
  \item torsion sheaves;
  \item \'etale topology for schemes;
\end{enumerate}
and in the case of manifolds and locally compact spaces we have Poincar\'e--Verdier duality:
\begin{enumerate}
  \item sheaves of abelian groups;
  \item locally compact spaces.
\end{enumerate}

In formalising the properties that are similar in each of these contexts we see that
\begin{itemize}
  \item we are considering ``image functors of sheaves'';
  \item we are using the monoidal structure of the category of sheaves.
\end{itemize}
In the general situation of abstract Verdier duality we can identify the following functors, for~$f\colon X\to Y$ a morphism in some category, and~$\mathcal{C}(X)$ some category of sheaves associated to~$X$.

\begin{center}
  \begin{tabular}{ccc}
    \toprule
    notation & name & signature \\\midrule
    $f^*$ & inverse image & $f^*\colon\mathcal{C}(Y)\to\mathcal{C}(X)$ \\
    $f_*$ & direct image & $f_*\colon\mathcal{C}(X)\to\mathcal{C}(Y)$ \\
    $f_!$ & exceptional (or proper, or twisted) direct image & $f_!\colon\mathcal{C}(X)\to\mathcal{C}(Y)$ \\
    $f^!$ & exceptional (or proper, or twisted) inverse image & $f^!\colon\mathcal{C}(Y)\to\mathcal{C}(X)$ \\
    $\HHom(-,-)$ & internal Hom & $\mathcal{C}(X)\times\mathcal{C}(X)\to\mathcal{C}(X)$ \\
    $-\otimes-$ & internal tensor product & $\mathcal{C}(X)\times\mathcal{C}(X)\to\mathcal{C}(X)$ \\
    \bottomrule
  \end{tabular}
\end{center}

We have relationships between these functors. These are the adjunctions
\begin{enumerate}
  \item $f^*\dashv f_*$;
  \item $f_!\dashv f^!$;
  \item $-\otimes C\dashv \HHom(C,-)$.
\end{enumerate}
In the situation of Grothendieck duality we moreover have the adjunction (with some abuse of notation, dropping~$\RRR$)
\begin{equation}
  f_*\dashv f^!
\end{equation}
if~$f\colon X\to Y$ is a proper map between the correct type of schemes. Or rather, we have~$f_*=f_!$ in this situation. But in general these are different, so we really have six functors and not just five. In the case of \'etale cohomology these six (or four) functors are also interesting to interpret, for which one is referred to \cite{milne-etale-cohomology}.

\subsection{Fourier--Mukai transforms}
Using this formalism of five (or six) functors we can get an interesting ``calculus of derived functors''. Some of these properties have been stated in the previous subsection, but we will now repeat them. The goal is to show that on the level of derived categories one gets lots of compatibilities which can be useful for computations, especially after we have discussed Orlov's existence result.

We will consider~$f\colon X\to Y$ a morphism of projective schemes over a field~$k$ which is the context of \cite{huybrechts-fourier-mukai-transforms}. The following are formal, in the sense that they are generalisations of the underived formulas.
\begin{description}
  \item[projection formula] {\ }
    
    $f$ proper, $\mathcal{F}^\bullet\in\derived^\bounded(\Coh/X)$ and~$\mathcal{G}^\bullet\in\derived^\bounded(\Coh/Y)$
    \begin{equation}
      \begin{tikzcd}
        \RRR f_*(\mathcal{F}^\bullet)\otimes^\LLL\mathcal{G}^\bullet \arrow{r}{\sim} & \RRR f_*\left( \mathcal{F}^\bullet\otimes^\LLL f^*(\mathcal{E}^\bullet) \right)
      \end{tikzcd}
    \end{equation}

  \item[$\LLL f^*$ and~$\otimes^\LLL$ commute] {\ }
    
    $\mathcal{F}^\bullet,\mathcal{G}^\bullet\in\derived^\bounded(\Coh/Y)$
    \begin{equation}
      \begin{tikzcd}
        \LLL f^*(\mathcal{F}^\bullet)\otimes^\LLL f^*(\mathcal{G}^\bullet) \arrow{r}{\sim} & \LLL f^*(\mathcal{F}^\bullet\otimes^\LLL\mathcal{G}^\bullet);
      \end{tikzcd}
    \end{equation}

  \item[$\LLL f^*$ and $\RRR f_*$ adjunction] {\ }
    
    $f$ projective, $\mathcal{F}^\bullet\in\derived^\bounded(\Coh/X)$, $\mathcal{G}^\bullet\in\derived^\bounded(\Coh/Y)$
    \begin{equation}
      \begin{tikzcd}
        \Hom_{\derived^\bounded(\Coh/X)}\left( \LLL f^*(\mathcal{G}^\bullet),\mathcal{F}^\bullet \right) \arrow{r}{\sim} & \Hom_{\derived^\bounded(\Coh/X)}\left( \mathcal{G}^\bullet,\RRR f_*(\mathcal{F}^\bullet) \right);
      \end{tikzcd}
    \end{equation}

  \item[$\otimes^\LLL$ and $\RRRHHom$ adjunction] {\ }
    
    $X$ smooth and projective, $\mathcal{F}^\bullet,\mathcal{G}^\bullet,\mathcal{H}^\bullet\in\derived^\bounded(\Coh/X)$
    \begin{equation}
      \begin{aligned}
        \RRRHHom(\mathcal{F}^\bullet,\mathcal{G}^\bullet)\otimes^\LLL\mathcal{G}^\bullet\cong\RRRHHom(\mathcal{F}^\bullet,\mathcal{G}^\bullet\otimes^\LLL\mathcal{H}^\bullet), \\
        \RRRHHom(\mathcal{F}^\bullet,\RRRHHom(\mathcal{G}^\bullet,\mathcal{H}^\bullet))\cong\RRRHHom(\mathcal{F}^\bullet\otimes^\LLL\mathcal{G}^\bullet,\mathcal{H}^\bullet), \\
        \RRRHHom(\mathcal{F}^\bullet,\mathcal{G}^\bullet\otimes^\LLL\mathcal{H}^\bullet)\cong\RRRHHom(\RRRHHom(\mathcal{G}^\bullet,\mathcal{F}^\bullet),\mathcal{H}^\bullet);
      \end{aligned}
    \end{equation}

  \item[global sections and $\RRRHHom$] {\ }
    
    $\mathcal{F}^\bullet\in\derived^\bounded(\Coh/X)$
    \begin{equation}
      \RRR\Gamma\circ\RRRHHom_X(\mathcal{F}^\bullet,-)=\RRRHom(\mathcal{F}^\bullet,-);
    \end{equation}

  \item[$\LLL f^*$ and $\RRRHHom$ commute] {\ }
    
    $\mathcal{F}^\bullet,\mathcal{G}^\bullet\in\derived^\bounded(Y)$
    \begin{equation}
      \begin{tikzcd}
        \LLL f^*\left( \RRRHHom_Y(\mathcal{F}^\bullet,\mathcal{G}^\bullet) \right) \arrow{r}{\sim} & \RRRHHom_X\left( \LLL f^*(\mathcal{F}^\bullet),\LLL f^*(\mathcal{G}^\bullet) \right);
      \end{tikzcd}
    \end{equation}

  \item[flat base change] {\ }

    for
    \begin{equation}
      \begin{tikzcd}
        Y' \arrow[swap]{d}{g} \arrow{r}{v} & Y \arrow{d}{f} \\
        X' \arrow[swap]{r}{u} & X
      \end{tikzcd}
    \end{equation}
    with~$f$ proper and~$u$ flat, $\mathcal{F}^\bullet\in\derived(\Qcoh/Y)$
    \begin{equation}
      \begin{tikzcd}
        u^*\circ\RRR f_*(\mathcal{F}^\bullet) \arrow{r}{\sim} & \RRR g_*\circ v^*(\mathcal{F}^\bullet).
      \end{tikzcd}
    \end{equation}
\end{description}

But as we saw in the description of the general six functor formalism we have another functor at our disposal. Or rather, in the context we are working in right now (smooth projective varieties over a scheme) we work with the dualising sheaf characterising the~$f^!$. If~$f\colon X\to Y$ is a morphism between such schemes we define explicitly
\begin{equation}
  \omega_f\coloneqq\omega_X\otimes f^*(\omega_Y^\vee)
\end{equation}
and
\begin{equation}
  \dim(f)\coloneqq\dim X-\dim Y.
\end{equation}
This means that~$f^!$ is given by
\begin{equation}
  f^!\colon\derived^\bounded(\Coh/Y)\to\derived^\bounded(\Coh/X):\mathcal{F}^\bullet\mapsto\LLL f^*(\mathcal{E}^\bullet)\otimes^\LLL\omega_f[\dim(f)].
\end{equation}
Then we get some new compatibilities, which are just a remanifestation of Grothendieck duality.
\begin{description}
  \item[Grothendieck duality] $\mathcal{F}^\bullet\in\derived^\bounded(X)$, $\mathcal{G}^\bullet\in\derived^\bounded(Y)$
    \begin{equation}
      \RRR f_*\left( \RRRHHom\left( \mathcal{F}^\bullet,\LLL f^*(\mathcal{G}^\bullet)\otimes^\LLL\omega_f[\dim(f)] \right) \right)\cong\RRRHHom\left( \RRR f_*(\mathcal{F}^\bullet),\mathcal{G}^\bullet \right);
    \end{equation}
  \item[$\RRR f_*$ and $f^!$ adjunction] $\mathcal{F}^\bullet\in\derived^\bounded(\Coh/X)$, $\mathcal{G}^\bullet\in\derived^\bounded(\Coh/Y)$
    \begin{equation}
      \begin{tikzcd}
        \Hom_{\derived^\bounded(\Coh/Y)}(\RRR f_*(\mathcal{F}^\bullet),\mathcal{G}^\bullet) \arrow{r}{\sim} & \Hom_{\derived^\bounded(\Coh/X)}(\mathcal{F}^\bullet,f^!(\mathcal{G}^\bullet)).
      \end{tikzcd}
    \end{equation}
\end{description}

The crux of all this is the following representability result.
\begin{theorem}
  \fixthis
  \label{theorem:bondal-orlov}
\end{theorem}
Because we can get strong results on Fourier--Mukai transforms (regardless of whether they are actually representing a functor as in the theorem) we have obtained an interesting ``calculus of derived functors''. This is an important area of current research, from many different perspectives.

Some of the results in \cite{huybrechts-fourier-mukai-transforms} which appeal immediately to Grothendieck duality are:
\begin{enumerate}
  \item an explicit formula for the left and right adjoint \cite[proposition 5.9]{huybrechts-fourier-mukai-transforms};
  \item braid group actions for spherical objects \cite[lemma 8.21]{huybrechts-fourier-mukai-transforms};
  \item the study of flips and flops \cite[\S 11.1]{huybrechts-fourier-mukai-transforms};
  \item semi-orthogonal decompositions of derived categories \cite[\S 11.2]{huybrechts-fourier-mukai-transforms} \expand{refer to this later};
  \item \ldots
\end{enumerate}


\subsection{The moduli of curves}
The paper that introduced \emph{stacks} to the world \cite{deligne-mumford-irreducibility-moduli-of-curves} also applies Grothendieck duality right from the start. The goal is to study the \emph{moduli space~$\mathcal{M}_g$ of curves of genus~$g$}, and show that it is irreducible, regardless of the choice of base field.

As they say themselves, the ``key definition of the whole paper'' is:
\begin{definition}
  Let~$S$ be any scheme. Let~$g\geq 2$. A \emph{stable curve of genus~$g$} over~$S$ is a proper flat morphism~$\pi\colon C\to S$ whose geometric fibres are reduced, connected, $1$\dash dimensional schemes~$C_s$ such that
  \begin{enumerate}
    \item $C_s$ has only ordinary double points;
    \item if~$E$ is a non-singular rational component of~$C_s$, then~$E$ meets the other components of~$C_s$ in more than~$2$ points;
    \item $\dim\HH^1(\mathcal{O}_{C_s})=g$.
  \end{enumerate}
\end{definition}
So two aspects of Grothendieck duality come to mind: the relative situation, and the (mild) singularities. We get a canonical invertible sheaf~$\omega_{C/S}$ on~$C$, which will act as a \emph{family} of sufficiently nice curves to connect any two points in the moduli space, thus proving irreducibility. One then proves the following properties of the dualising sheaf:
\begin{enumerate}
  \item $\omega_{C/S}^{\otimes n}$ is relatively very ample for~$n\geq 3$;
  \item $\pi_*(\omega_{C/S}^{\otimes n})$ is locally free of rank~$(2n-1)(g-1)$.
\end{enumerate}
The proof of these properties uses the fact that we ``almost'' get a smooth curve of genus~$g$, and we study the different irreducible components, together with the explicit manifestation of Grothendieck duality for curves with at most ordinary double points.

Hence we can conclude that, taking~$n=3$, we can realise a stable curve~$C\to S$ as a family of curves inside~$\mathbb{P}^{5g-6}$ such that the Hilbert polynomial of each point is~$(6n-1)(g-1)$.

This yields the construction of a subscheme~$\mathrm{H}_g\subseteq\mathrm{Hilb}_{\mathbb{P}^{5g-6}}^{(6n-1)(g-1)}$ of \emph{tricanonically embedded stable curves}, i.e.\ the functor described by
\begin{equation}
  \Hom_{\mathrm{Sch}}(S,\mathrm{H}_g)\cong\left\{ \left( \text{$\pi\colon C\to S$ stable; $\Proj\left( \pi_*(\omega_{C/S}^{\otimes 3}) \right)\cong\mathbb{P}_S^{5g-6}$} \right) \right\}/{\cong}
\end{equation}
for a scheme~$S$. By taking the quotient of the (open locus of smooth curves of the) scheme~$\mathrm{H}_g$ by the~$\mathrm{PGL}_{5g-6}$\dash action we obtain a model for the moduli space of (smooth) curves, and hence we can try to compute things.

From this point on the proof does not use Grothendieck duality anymore, so I will end the summary here.


\subsection{Other applications}
Each of the following applications more than deserves a proper treatment. Unfortunately this is not possible here, due to lack of time, space and familiarity with the subject. They are here to show how diverse applications of Grothendieck duality can get. Any error in this list is due to my limited knowledge on the subject.

\paragraph{Local duality}
The study of local rings and singularities leads to working with Cohen--Macaulay rings and modules, and understanding these in as concrete terms as possible. It is related to representation theory as well.

\paragraph{Singularity categories}
This is another approach to studying singularities, now in a more global setting. It is similar to the previous application in some respects, but more alike studying Fourier--Mukai transforms and derived categories in others.

\paragraph{Noncommutative algebra}
The notion of dualising complex has a counterpart for noncommutative rings.

\paragraph{Noncommutative algebraic geometry}
The notion of Serre and Grothendieck duality leads to studying abstract Serre functors in triangulated or dg~categories. This is also related to Calabi--Yau categories.

\paragraph{Arithmetic geometry}
The relative formalism also applies to arithmetic geometry, for example in studying Eisenstein ideals \cite{mazur-modular-curves-eisenstein-ideal}. I know absolutely nothing about it.


\section{Hartshorne's proof: dualising and residual complexes}
\label{section:hartshorne}
\subsection{Context}
\label{subsection:context}
The first proof of Grothendieck duality was given by Robin Hartshorne in 1966 \cite{hartshorne-residues-and-duality}, based on notes provided by Alexander Grothendieck in 1963. As the statement and proof require the use of derived categories, Jean--Louis Verdier's ongoing work was included, and it was (as far as I can tell) the first published treatise of derived categories.

\paragraph{Issues with the proof}
This approach is the most geometric of them all, but also the most complicated. To quote Amnon Neeman \cite{neeman-grothendieck-duality-bousfield-brown}:
\begin{quote}
  [\ldots] Since derived categories are basically unsuited for local computations, the argument turns out to be quite unpleasant.
\end{quote}
If one reads the proof as outlined in \cite{hartshorne-residues-and-duality} this will become clear: after introducing the required notions of derived categories in algebraic geometry (the first 100 pages) the proof takes 250 pages. These 250 pages also only summarise many important results on local cohomology and depend heavily on technical results in the EGA's.

Moreover, the proof from \cite{hartshorne-residues-and-duality} is incomplete, and contains errors. Regarding the incompleteness the author himself says in \cite[\S II.5]{hartshorne-residues-and-duality}:
\begin{quote}
  Now these examples are only three of many more compatibilities which will come immediately to the reader's mind. I could make a big list, and in principle could prove each one on the list. [\ldots] And since the chore of inventing these diagrams and checking their commutativity is almost mechanical, the reader would not want to read them, nor I write them. [\ldots]
\end{quote}
Hence the reader is left with checking lots and lots of commutative diagrams, some of them depending on very subtle sign conventions in homological algebra!

\paragraph{Trace maps and base change}
There is an important omission from the proof: the compatibility of the trace map for smooth morphisms with arbitrary base change. If~$f\colon X\to Y$ is a proper, surjective, smooth map of schemes whose fibers are equidimensional of dimension~$n$, then we had the \emph{trace map} \cite[\S VII.4]{hartshorne-residues-and-duality}
\begin{equation}
  \gamma_f\colon\RR^nf_*(\omega_{X/Y})\to\mathcal{O}_Y
\end{equation}
which is an isomorphism if~$f$ has geometrically connected fibers. Now let
\begin{equation}
  \begin{tikzcd}
    X' \arrow{r}{v} \arrow{d}{g} & X \arrow{d}{f} \\
    Y' \arrow{r}{u} & Y
  \end{tikzcd}
\end{equation}
be a cartesian diagram, then we get an isomorphism
\begin{equation}
  u^*\left( \RR^nf_*(\omega_{X/Y}) \right)\to\RR^ng_*\left( v^*(\omega_{X/Y}) \right)\cong\RR^ng_*(\omega_{X'/Y'}).
\end{equation}
The desired compatibility then asserts that
\begin{equation}
  \begin{tikzcd}
    u^*\left( \RR^nf_*(\omega_{X/Y}) \right) \arrow{rr}{\cong} \arrow{rd}{u^*(\gamma_f)} & & \RR^ng_*(\omega_{X'/Y'}) \arrow{ld}{\gamma_{g}} \\
    & u^*(\mathcal{O}_Y)=\mathcal{O}_{Y'}
  \end{tikzcd}
\end{equation}
is a commutative diagram. A nice discussion of the state of this base change compatibility can be found in \cite[\S 1.1]{conrad-grothendieck-duality-and-base-change}. To summarise: the proof is left to the reader in \cite[\S VII.4]{hartshorne-residues-and-duality}, and its proof is highly non-trivial, which brings us by \cite{conrad-grothendieck-duality-and-base-change}.

\paragraph{Companion to the proof}
This book is written as a complement to the original proof, providing information on the omissions and fixing the numerous mistakes in the original proof. As the theory of derived categories was in its infancy, many things were still unclear, and this caused errors. Some of these are trivial to fix, others are severe and require a completely different proof. And to top things of: some (minor) mistakes have been found in Brian Conrad's book, but these don't require difficult fixes and a detailed erratum is available.

In total, one understands that if the proof and a companion to the proof take about 500 pages, it's hardly an easy proof.

\subsection{Outline of the proof}
Summarising 500 pages of proof is a rather non-trivial task to do, but I will try to outline my view on the proof and its structure. Hopefully this helps in tackling the proof, and identifying which parts could be of interest to the reader. I will do this by summarising the different chapters, later on some interesting points will be highlighted in separate paragraphs.

As mentioned before (but maybe not explicitly enough): this approach to Grothendieck duality can be summarised by the following slogan.
\begin{quote}
  We define~$f^!$ by looking for a dualising complex and \emph{defining} the functor in terms of this complex.
\end{quote}
The whole setup of the book should be considered in this point of view.

\begin{description}
  \item[chapters 1 and 2] As derived categories were still in their infancy and there was not a published text available about them, they are first introduced and then applied to the situation of schemes. For a more up-to-date introduction one can look at \cite[chapters 1--3]{huybrechts-fourier-mukai-transforms}.
  \item[chapter 3] The proof of Grothendieck duality for projective morphisms. In this ``easy'' case we can do more explicit computations, and control the dualising object. The idea is to factor sufficiently nice morphisms (see \cref{subsection:embeddable-morphisms}) into
    \begin{enumerate}
      \item smooth morphisms,
      \item finite morphisms;
    \end{enumerate}
    and introduce the functor~$f^!$ for each of these. The functor~$f^!$ for a finite morphism is denoted~$f^\flat$, the one for a smooth morphism is~$f^\sharp$.
    
    By checking compatibility of these two definitions (which are suggested by the Ideal theorem, as given in the previous lecture) one obtains a theory of~$f^!$ for these nice morphisms \cite[theorem III.8.7]{hartshorne-residues-and-duality} (but not all the required properties for~$f^!$). This is then used to obtain Grothendieck duality for \emph{projective} morphisms \cite[\S III.9, III.10]{hartshorne-residues-and-duality}.

    Remark that \cite{altman-kleiman-grothendieck-duality} looks like it proves Grothendieck duality in this projective case. But actually it only proves a statement similar to Serre duality (in the original sense of \cite{serre-faisceaux-algebriques-coherents}), no derived categories are mentioned: the proof uses spectral sequences and some results on depth to obtain a statement for singular varieties over a field. The book itself remains interesting (for historical reasons) as it is the earliest easy-to-read text on the algebraic aspects of algebraic geometry I have seen so far.

    A similar idea of factoring morphisms into tractable ones occurs in \cref{section:deligne}.

  \item[chapter 4] As discussed in the section on applications of Grothendieck duality there is an interesting notion of local duality, related to local cohomology. Before tackling global Grothendieck duality one has to understand what happens in the local case, as this is what is used to characterise the objects defined in the next chapters.

    Historically, local duality is treated in \cite{sga2} and \cite{hartshorne-local-cohomology}. Remark that this second book are the notes of a seminar by Hartshorne on local duality, and are originally from 1961, the same period as SGA2 (which was done in 1961--1962).

  \item[chapter 5] Recall that this approach to Grothendieck duality can be summarised by the following slogan.
    \begin{quote}
      We define~$f^!$ by looking for a dualising complex and \emph{define} the functor in terms of this complex.
    \end{quote}
    But the statement of Grothendieck duality doesn't mention dualising complexes explicitly. Hence it is not required to develop this machinery if one is looking for~$f^!$, but it does give an explicit flavour to it. In this chapter the machinery and properties of dualising complexes are discussed. The goal is to understand how dualising complexes relate to local duality, how this behaves with respect to singularities and how we can interpret the dualising complexes. Some of these properties are discussed in \cref{subsection:dualising-complexes}.

  \item[chapter 6] Unfortunately, dualising complexes live in a derived category, and this is a non-local object\addreference. To solve this problem \emph{residual complexes} are introduced. These are a manifestation of dualising complexes in the non-derived category of chain complexes. A nice motivation for having a theory for both dualising and residual complexes is given on \cite[pages 106--107]{conrad-grothendieck-duality-and-base-change}. In the case of a curve~$C$ these two manifestations are
    \begin{enumerate}
      \item the dualising complex $\Omega_{C/k}^1[1]$ (familiar from Riemann--Roch!);
      \item the residual complex
        \begin{equation}
          \dotso\to i_{\xi,*}\left( \Omega_{X/k,\xi}^1 \right)
        \end{equation}
        where~$\xi$ is the generic point of~$C$.
    \end{enumerate}
    These complexes are quasi-isomorphic to eachother, but the residual complex is a bunch of injective hulls taken together, which can be taken to live in a non-derived category and still allow for computations. For more information on the definition see \cref{subsection:residual-complexes}.

    Now a theory for embeddable morphisms and residual complexes is developed, using functors~$f^\yy$ and~$f^\zz$ for finite and smooth morphisms\footnote{This is really weird notation.}. Their definitions depend on (pointwise) dualising complexes, but they are truly functors on the non-derived level. This chapter is the technical part of the book, which lots of compatibility checks.

  \item[chapter 7] In this chapter Grothendieck duality in its general form is finally proved. We have obtained many preliminary results, and this allows us to summarise the final proof \cite[theorem VII.3.3]{hartshorne-residues-and-duality} as follows:
    \begin{enumerate}
      \item Grothendieck duality for~$\RRRHHom$ is local, hence we reduce~$Y$ to the spectrum of a local ring (so the base is affine).
      \item By some machinery of derived categories we can replace the complex by a single quasicoherent sheaf on~$X$.
      \item We can replace the quasicoherent sheaf on~$X$ by a coherent one using a direct limit argument.
      \item As~$Y$ is affine the local statement for~$\RRRHHom$ becomes a global statement for~$\RRRHom$.
      \item We check compatibility of the global statement with composition of two morphisms (this is not a part of the conceptual flow of the proof in my opinion). This is where residual complexes are required. One of the compatibilities requires a coherence condition, which explains the reduction in the third step.
      \item Using noetherian induction on~$X$ we can assume that the theorem is proven for every~$g\colon Z\to Y$ where~$i\colon Z\to X$ is a closed immersion with~$Z\neq X$ and~$g=f\circ i$.
      \item As~$X$ is proper over~$Y$ we apply Chow's lemma to find an~$X'$ which is projective over~$Y$ and a projective morphism~$g\colon X\to X'$ which is an isomorphism over some non-empty open subset~$U$. By the noetherian induction we can assume the theorem proven for the complement, which allows us to reduce the statement to the projective morphism in the factorisation of Chow's lemma.
      \item Now we can apply the results we had for projective morphisms, and conclude.
    \end{enumerate}
    The remainder of the chapter is dedicated to spelling out the trace map and making the duality result more explicit for proper smooth morphisms.
\end{description}

\subsection{Embeddable morphisms}
\label{subsection:embeddable-morphisms}
To proof Grothendieck duality one first proves it for embeddable morphisms.
\begin{definition}
  Let~$S$ be a base scheme. A morphism~$f\colon X\to Y$ is \emph{embeddable} (over~$S$) if there exists a smooth scheme~$P$ (over~$S$) and a finite morphism~$i\colon X\to P_Y=P\times_SY$ such that~$f=p_2\circ i$.
\end{definition}
Exactly how useful is this definition?
\begin{enumerate}
  \item If~$f\colon X\to Y$ is finite then~$f$ can be factored through~$P=S$.
  \item If~$f\colon X\to Y$ is projective, where~$Y$ is quasicompact and admits an ample sheaf, then~$f$ can be factored through some~$P=\mathbb{P}_Y^n$ \cite[II.5.5.4(ii)]{egaII}.
\end{enumerate}
The main issue is that morphisms of finite type (a very general class of morphisms) are locally embeddable, but not globally so. Hence this approach does not yield a theory of~$f^!$ in general. To overcome this issue we need the notion of dualising and especially residual complexes.

\subsection{Dualising complexes}
\label{subsection:dualising-complexes}
Recall from the description of duality on~$\Spec\mathbb{Z}$ (see the previous lecture) that we had a ``bounded complex of quasicoherent sheaves with coherent cohomology''. In this case all the highbrow terminology boils down to ``a bounded complex of abelian groups with finitely generated cohomology''. As the complex we considered was an injective resolution of~$\mathbb{Z}$ this condition is clearly satisfied.

By considering this particular complex we obtained a duality functor
\begin{equation}
  \dual\colon M^\bullet\to\RRRHom^\bullet(M^\bullet,\mathbb{Z})
\end{equation}
for~$M^\bullet$ an object in~$\derived_{\mathrm{fg}}^\bounded(\Ab)=\derived_{\coh}^\bounded(\Spec\mathbb{Z})$, i.e.\ applying the dual twice yields something functorially isomorphic to what you started with. This is completely analogous to the case of vectorspaces: one has to start with a finite-dimensional one to get the double dual isomorphic to the original one.

Generalising this we get to the following definition.
\begin{definition}
  Let~$X$ be locally noetherian. A complex~$\mathcal{R}^\bullet\in\derived_\coh^+(X)_\fid$ such that for each~$\mathcal{F}^\bullet\in\derived(X)$ the morphism
  \begin{equation}
    \eta\colon\mathcal{F}^\bullet\to\ddual\circ\ddual(\mathcal{F}^\bullet)=\RRRHHom^\bullet\left( \RRRHHom(\mathcal{F}^\bullet,\mathcal{R}^\bullet),\mathcal{R}^\bullet \right)
  \end{equation}
  is an isomorphism.
\end{definition}
As is fashionable in algebraic geometry, we have turned our problem into a definition. But this is a remarkably interesting definition, as we can observe the following \emph{properties}:
\begin{enumerate}
  \item the complex~$\mathcal{R}^\bullet$ is quasi-isomorphic to a bounded complex of quasicoherent injective sheaves, hence we get that~$\ddual\colon\derived(X)\to\derived(X)$ sends~$\derived_\coh^\bounded(X)$ to itself, and it interchanges~$\derived_\coh^+(X)$ and~$\derived_\coh^-(X)$;
  \item if~$X$ is regular of finite Krull dimension then~$\mathcal{O}_X$ is already a dualising complex \cite[example V.2.2]{hartshorne-residues-and-duality};
  \item one can check whether~$\mathcal{R}^\bullet$ is dualising at all the stalks of closed points of~$X$ \cite[corollary V.2.3]{hartshorne-residues-and-duality};
  \item dualising complexes are preserved by~$f^\flat$ and~$f^\sharp$ \cite[proposition V.2.4 and theorem V.8.3]{hartshorne-residues-and-duality}, hence for embeddable~$f$ we get that~$f^!$ preserves dualising complexes, and this means that we can compute~$f^!$ by the following isomorphism
    \begin{equation}
      f^!(\mathcal{F}^\bullet)\cong\ddual_X\circ\LLL f^*\circ\ddual_Y(\mathcal{F}^\bullet)
    \end{equation}
    with
    \begin{equation}
      \begin{aligned}
        \ddual_X(-)&\coloneqq\RRRHHom_X^\bullet(-,f^!(\mathcal{R}^\bullet)) \\
        \ddual_Y(-)&\coloneqq\RRRHHom_Y^\bullet(-,\mathcal{R}^\bullet)
      \end{aligned}
    \end{equation}
    if~$\mathcal{R}^\bullet$ is a dualising complex on~$Y$ and~$f\colon X\to Y$ is embeddable \cite[proposition V.8.5]{hartshorne-residues-and-duality};
  \item if more generally~$f$ is of finite type with~$Y$ noetherian and~$\mathcal{R}^\bullet$ a dualising complex on~$Y$ then~$f^!(\mathcal{R}^\bullet)$ will be one on~$X$;
  \item dualising complexes are unique up to tensoring with invertible sheaves and shifts \cite[theorem V.3.1]{hartshorne-residues-and-duality}.
\end{enumerate}
Regarding the question of its \emph{existence}, one has the following necessary and sufficient conditions \cite[\S V.10]{hartshorne-residues-and-duality}:
\paragraph{Sufficient conditions}
Hence, under which conditions can we prove the existence of a dualising complex?
\begin{enumerate}
  \item $X$ Gorenstein and of finite Krull dimension.
  \item $X$ of finite type over~$Y$ with~$Y$ admitting a dualising comlex (which has $X$ of finite type over a field~$k$ as a special case, hence we get Serre duality for arbitrary singularities).
\end{enumerate}
\paragraph{Necessary conditions}
Hence, what properties of~$X$ does the existence of a dualising complex imply?
\begin{enumerate}
  \item $X$ has finite Krull dimension.
  \item $X$ is catenary.
\end{enumerate}
As the proof of Grothendieck duality by constructing an explicit formula for~$f^!$ depends on dualising complexes, one could hope for a more general result by taking a different approach. This will be discussed later on. First we have to discuss how one can use local duality and dualising complexes to obtain a global theory.

\subsection{Residual complexes}
\label{subsection:residual-complexes}
The problem is that we cannot glue dualising complexes together, as the derived category is not a local object \addreference. Hence we need to do some work to use our dualising complexes from the previous paragraph and obtain~$f^!$.

The idea is to take a dualising complex~$\mathcal{R}^\bullet\in\derived_\coh^+(X)$ and turn it into an actual complex (i.e.\ in some~$\Ch(X)$) which will be called the \emph{residual complex}\expand{why this terminology?}.

As we can glue actual complexes together, we can obtain a~$f^!$ by gluing residual complexes together. Of course, we need to know that it doesn't matter whether we use dualising complexes or residual complexes. I.e.\ we will look for a functor~$\cousin\colon\derived_\coh^+(X)\to\Ch_\coh^+(\Qcoh_\inj/X)$ such that~$\quotient\circ\cousin(\mathcal{R}^\bullet)\cong\mathcal{R}^\bullet$, where~$\quotient$ is the quotient functor in the construction of the derived category, and~$\Ch_\coh^+(\Qcoh_\inj/X)$ is a model for~$\derived_\coh^+(X)$.

The definition of a residual complex seems odd at first sight.
\begin{definition}
  Let~$X$ be a locally noetherian prescheme. A \emph{residual complex}~$\mathcal{K}^\bullet$ on~$X$ is a bounded below complex of quasicoherent injective~$\mathcal{O}_X$\dash modules with coherent cohomology, together with an isomorphism
  \begin{equation}
    \bigoplus_{p\in\mathbb{Z}}\cong\bigoplus_{x\in X}\mathrm{J}(x)
  \end{equation}
  where~$\mathrm{J}(x)$ is the quasicoherent injective~$\mathcal{O}_X$\dash module given by the constant sheaf with values in an injective hull of~$k(x)$ over~$\mathcal{O}_{X,x}$ on~$\mathrm{cl}(\{x\})$, and zero elsewhere.
\end{definition}
The reason why this definition is interesting can be deduced from \cite[proposition V.3.4]{hartshorne-residues-and-duality}, which gives a description of dualising complexes in the stalk. This description is given by a purity result for Ext-functors, hence considering these rather special residual complexes which are constructed from injective hulls makes sense.

The functor~$\cousin$ uses the theory of \emph{Cousin complexes}, and this is based on suitable filtrations of~$X$. For more information, see \cite[chapter IV]{hartshorne-residues-and-duality}.

This ``equivalence'' of residual and dualising complexes is one of the subtle points in the proof, and things are not correct the way they are stated. For a discussion of the problems, and a solution, see the discussion around \cite[lemma 3.2.1]{conrad-grothendieck-duality-and-base-change}.

As the equivalence is true under reasonable hypotheses, one can then define~$f^!$ along similar means using Nagata compactifications. This is where the main technical part of the proof is found \cite[\S VI.2--VI.5]{hartshorne-residues-and-duality}. It requires checking lots and lots of commutative diagrams, and this is one of the reasons for the existence of \cite{conrad-grothendieck-duality-and-base-change}.


\section{Deligne's proof: the right adjoint}
\label{section:deligne}
If one wishes to settle for a Grothendieck duality theory, without dualising and residual complexes, it is possible to prove the existence of the right adjoint~$f^!$ by other means \cite{deligne-appendix-f-upper-shriek,verdier-base-change-twisted-inverse-image}. One can then show that the remaining aspects of Grothendieck duality follow from the existence of this adjoint.

The idea for this approach comes from Verdier duality, which is a generalised Poincar\'e duality for topological spaces. It is possible to obtain the results in an almost formal way, if one has a good theory of ``cohomology with proper support''.

\subsection{Nagata's compactification theorem}
The main idea in Deligne's approach is to replace the morphism~$f\colon X\to Y$ by more tractable ones. As in the case of general topology it is often easier to prove something for compact spaces. The notion of compactness (in the usual sense, often denoted quasicompactness) is only mildly interesting, and does not suffice to prove Grothendieck duality. The correct notion of compactness is properness. So we make the following definition.
\begin{definition}
  Let~$f\colon X\to Y$ be a morphism of schemes. It is \emph{compactifiable} (in the terminology of \cite[appendix]{hartshorne-residues-and-duality}) if there exist morphisms~$g\colon X\to\overline{X}$ and~$h\colon\overline{X}\to Y$ such that
  \begin{equation}
    \begin{tikzcd}
      X \arrow{rr}{f} \arrow{rd}{g} & & Y \\
      & \overline{X} \arrow{ur}{h}
    \end{tikzcd}
  \end{equation}
  such that
  \begin{enumerate}
    \item $g$ is an open immersion;
    \item $h$ is proper.
  \end{enumerate}
\end{definition}
The question now becomes: which morphisms are compactifiable? The answer is very interesting, and given by the following theorem \cite{nagata-imbedding,nagata-generalization-imbedding,conrad-delignes-notes-nagata-compactification}.
\begin{theorem}[Nagata's compactification theorem]
  \label{theorem:nagata}
  Let~$f\colon X\to Y$ be separated and of finite type between quasicompact and quasiseperated schemes. Then~$f$ is compactifiable.
\end{theorem}
Hence this suffices to obtain the existence of~$f^!$ in a very general context, e.g.\ noetherian schemes.

Remark that the history of this theorem is intriguing: it was proved in 1962, but in the language of Zariski--Riemann spaces and valuations, which is algebraic geometry in the sense of Zariski and Weil. Hence the proof nor the statement were known in the language of schemes. This has now changed \cite{lutkebohmert-compactification,conrad-delignes-notes-nagata-compactification,deligne-plongement-de-nagata}. This result would be an interesting topic for another series of lectures in the seminar.

\subsection{Outline of the proof}
Using this notion of compactifiability he defines a functor~$f_!$ (or~$\RRR f_!$, depending on the notation, but it is \emph{not} a derived functor) which is related to~$\RRR f_*$. Then using a result by Verdier \cite{verdier-bourbaki-300} we get a right adjoint~$f^!$ for~$\RRR f_*$. Then he proves that~$f_!$ and~$f^!$ are adjoint to eachother (in the context of pro-objects), and deduces some properties. For more deductions on the properties of~$f^!$ see \cite{verdier-base-change-twisted-inverse-image}.

A similar approach is by the way taken to define~$f_!$ and~$f^!$ in the context of \'etale cohomology \cite[expos\'e XVII]{sga43}.

\section{Neeman's proof: Brown's representability theorem}
\label{section:neeman}
\subsection{Existence of adjoint functors}

\subsection{Application: Grothendieck duality}

\section{Murfet's proof: The mock homotopy category of projectives}
\label{section:murfet}
\subsection{Definitions}
\label{subsection:definitions}

\subsection{Proof}
\label{subsection:proof}

\section{Other proofs}
\label{section:other}
\subsection{Rigid dualising complexes}
\label{subsection:yekutieli-zhang}

\subsection{Pseudo-coherent complexes}
\label{subsection:lipman}
% figure out whether what I'm saying here is correct

% are there more?


\printbibliography

\end{document}

