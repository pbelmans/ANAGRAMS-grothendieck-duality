\documentclass[10pt,a4paper]{article}
\usepackage{hyperref}
\usepackage{cleveref}
\hypersetup{hypertexnames = false, bookmarksdepth = 2, bookmarksopen = true, colorlinks, linkcolor = black, citecolor = black, urlcolor = black, pdfstartview={XYZ null null 1}}

\usepackage{amsfonts}
\usepackage[fleqn, leqno]{amsmath}
\usepackage{amsthm}
\usepackage{biblatex}
\usepackage{booktabs}
\usepackage{diagbox}
\usepackage{enumitem}
\usepackage{fixltx2e}
\usepackage{mathtools}
\usepackage{thmtools}
\usepackage{tikz-cd}
\usepackage[colorinlistoftodos]{todonotes}
\usepackage{xparse}
\usepackage{xspace}

\usepackage[T1]{fontenc}
\usepackage[charter]{mathdesign}
\usepackage[scaled]{beramono,berasans}
\usepackage{eucal}
\usepackage{epstopdf}
\usepackage{microtype}
\frenchspacing

\addbibresource{bibliography.bib}

\addtolength\parskip{.4ex}
\setlength\parindent{0cm}

\relpenalty=10000
\binoppenalty=10000

% todonotes configuration
\newcounter{todocounter}
\DeclareDocumentCommand\addreference{g}{\stepcounter{todocounter}\todo[color = blue!30, fancyline]{\thetodocounter. Add reference\IfNoValueF{#1}{: #1}}\xspace}
\DeclareDocumentCommand\checkthis{g}{\stepcounter{todocounter}\todo[color = red!50, fancyline]{\thetodocounter. Check this\IfNoValueF{#1}{: #1}}\xspace}
\DeclareDocumentCommand\fixthis{g}{\stepcounter{todocounter}\todo[color = orange!50, fancyline]{\thetodocounter. Fix this\IfNoValueF{#1}{: #1}}\xspace}
\DeclareDocumentCommand\expand{g}{\stepcounter{todocounter}\todo[color = green!50, fancyline]{\thetodocounter. Expand\IfNoValueF{#1}{: #1}}\xspace}
\newcommand\removethis{\stepcounter{todocounter}\todo[color=yellow!50]{\thetodocounter. Remove this?}}

% environments
\declaretheoremstyle[
  spaceabove = 3pt,
  spacebelow = 3pt,
]{lecture}
\theoremstyle{lecture}
\newtheorem{theorem}{Theorem}
\newtheorem{corollary}[theorem]{Corollary}
\newtheorem{definition}[theorem]{Definition}
\newtheorem{example}[theorem]{Example}
\newtheorem{lemma}[theorem]{Lemma}
\newtheorem{proposition}[theorem]{Proposition}
\newtheorem{remark}[theorem]{Remark}


\mathchardef\mhyphen="2D
\newcommand\dash{\nobreakdash-\hspace{0pt}}
\newcommand\bounded{\ensuremath{\mathrm{b}}}
\newcommand\Coh{\ensuremath{\mathrm{Coh}}}
\newcommand\dd{\mathrm{d}}
\newcommand\derived{\ensuremath{\mathbf{D}}}
\newcommand\KKK{\ensuremath{\mathbf{K}}}
\newcommand\Qcoh{\ensuremath{\mathrm{Qcoh}}}
\newcommand\RRR{\ensuremath{\mathbf{R}}}

\DeclareMathOperator\Ch{Ch}
\DeclareMathOperator\Ext{Ext}
\DeclareMathOperator\hh{h}
\DeclareMathOperator\HH{H}
\DeclareMathOperator\Hom{Hom}
\DeclareMathOperator\tr{tr}
\DeclareMathOperator\Proj{Proj}
\DeclareMathOperator\res{res}
\DeclareMathOperator\Spec{Spec}
\DeclareMathOperator\supp{supp}


\title{Grothendieck duality: lecture 3 \\[.2em] \Large Grothendieck duality and applications}
\author{Pieter Belmans}
\date{January 22, 2014}

\begin{document}
\maketitle

\begin{abstract}
  
\end{abstract}

\tableofcontents

\clearpage

\section{Grothendieck duality}
\label{section:grothendieck-duality}
\subsection{Motivation}
\label{subsection:motivation}

\subsection{The ideal theorem}
The first thing we can consider as a form of Grothendieck duality is \cite[Ideal theorem on page 6]{hartshorne-residues-and-duality}. This summarises what one tries to prove to be able to speak of a ``Grothendieck duality result''. After the statement we will collect some contexts in which we can prove this.
\begin{theorem}[Ideal theorem] {\ }
  \label{theorem:classical}
  \begin{enumerate}
    \item\label{enumerate:classical-a} For every morphism $f\colon X\to Y$ of finite type of preschemes, there is a functor
      \begin{equation}
        f^!\colon\derived(Y)\to\derived(X)
      \end{equation}
      such that
      \begin{enumerate}
        \item\label{enumerate:classical-a-1} if $g\colon Y\to Z$ is a second morphism of finite type, then $(g\circ f)^!=f^!\circ g^!$;
        \item\label{enumerate:classical-a-2} if $f$ is a smooth morphism, then
          \begin{equation}
            f^!(\mathcal{G})=f^*(\mathcal{G})\otimes\omega,
          \end{equation}
          where $\omega=\Omega_{X/Y}^n$ is the sheaf of highest order differentials;
        \item\label{enumerate:classical-a-3} if $f$ is a finite morphism, then
          \begin{equation}
            f^!(\mathcal{G})=\HHom_{\mathcal{O}_Y}(f_*\mathcal{O}_X,\mathcal{G})^\sim.
          \end{equation}
      \end{enumerate}
    \item\label{enumerate:classical-b} For every proper morphism $f\colon X\to Y$ of preschemes, there is a \emph{trace morphism}
      \begin{equation}
        \Tr_f\colon\RRR f_*\circ f^!\Rightarrow\identity
      \end{equation}
      of functors from $\derived(Y)$ to $\derived(X)$ such that
      \begin{enumerate}
        \item\label{enumerate:classical-b-1} if $g\colon Y\to Z$ is a second proper morphism, then $\Tr_{g\circ f}=\Tr_g\circ\Tr_f$;
        \item\label{enumerate:classical-b-2} if $X=\mathbb{P}_Y^n$, then $\Tr_f$ is the map deduced from the canonical isomorphism $\mathbf{R}^nf_*(\omega)\cong\mathcal{O}_Y$;
        \item\label{enumerate:classical-b-3} if $f$ is a finite morphism, then $\Tr_f$ is obtained from the natural map ``evaluation at one''
          \begin{equation}
            \HHom_{\mathcal{O}_Y}(f_*\mathcal{O}_X,\mathcal{G})\to\mathcal{G}.
          \end{equation}
      \end{enumerate}
    \item\label{enumerate:classical-c} If $f\colon X\to Y$ is a proper morphism, then the duality morphism
      \begin{equation}
        \Theta_f\colon\RRRHom_X(\mathcal{F},f^!(\mathcal{G}))\to\RRRHom_Y(\mathbf{R}f_*\mathcal{F},\mathcal{G})
      \end{equation}
      obtained by composing the natural map\footnote{Obtained as the Yoneda pairing, see \cite[page 5]{hartshorne-residues-and-duality}.} above with $\Tr_f$, is an isomorphism for $\mathcal{F}\in\derived(X)$ and $\mathcal{G}\in\derived(Y)$.
  \end{enumerate}
\end{theorem}
The question is: in which situations can we prove this ideal theorem? The answer (as per \cite{hartshorne-residues-and-duality}, nowadays it is more general) is:
\begin{enumerate}
  \item noetherian schemes of finite Krull dimensions and morphisms which factor through a suitable projective space \cite[\S III.8, \S III.10, \S III.11]{hartshorne-residues-and-duality}: all statements are applied for $\derived_\qc^+(-)$ except the last in which case $\mathcal{F}$ lives in $\derived_\qc^-(X)$;
  \item noetherian schemes which admit dualizing complexes (see \cite[\S V.10]{hartshorne-residues-and-duality}, it implies finite Krull dimension) and morphisms whose fibres are of bounded dimension \cite[\S VII.3]{hartshorne-residues-and-duality}: all statements are applied to $\derived_\coh^+(-)$ except the last in which case $\mathcal{F}$ lives in $\derived_\qc^-(X)$;
  \item noetherian schemes of finite Krull dimension and smooth morphisms \cite[\S VII.4]{hartshorne-residues-and-duality}: the results in \ref{enumerate:classical-a} are applied to $\derived_\qc^+(-)$, in \ref{enumerate:classical-b} to $\derived_\qc^\bounded(-)$ and in \ref{enumerate:classical-c} to $\mathcal{F}$ in $\derived_\qc^-(X)$ and $\mathcal{G}$ in $\derived_\qc^\bounded(Y)$;
  \item noetherian schemes and arbitrary morphisms \cite[appendix]{hartshorne-residues-and-duality}, but only statements \ref{enumerate:classical-a-1}, \ref{enumerate:classical-b-1} and \ref{enumerate:classical-c}: the results are applied to $\derived(\Qcoh(-))$.
\end{enumerate}

\begin{table}
  \centering
  \begin{tabular}{cccc}
    \toprule
    situation & $X$ and $Y$                           & $f$                             \\\midrule
    1         & noetherian, finite Krull dimension    & factor through $\mathbb{P}_Y^n$ \\
    2         & noetherian with dualizing complex  \\
    3         & noetherian, finite Krull dimension    & smooth\\
    4         & noetherian \\
    \bottomrule
  \end{tabular}
  \caption{Description of the 4 situations for Grothendieck duality in \cite{hartshorne-residues-and-duality}}
  \label{table:situations}
\end{table}

\begin{table}
  \centering
  \begin{tabular}{rccc}
    \toprule
    property           & situation & $\derived(X)$       & $\derived(Y)$ \\\midrule
    existence of $f^!$ & 1         &                     & $\derived_\qc^+(Y)$ \\
                       & 2         &                     & $\derived_\coh^+(Y)$ \\
                       & 3         &                     & $\derived_\qc^\bounded(Y)$ \\
                       & 4         &                     & $\derived^+(\Qcoh/Y)$ \\
    trace morphism     & 1         &                     & $\derived_\qc^+(Y)$ \\
                       & 2         &                     & $\derived_\coh^+(Y)$ \\
                       & 3         &                     & $\derived_\qc^\bounded(Y)$ \\
                       & 4         &                     & $\derived^+(\Qcoh/Y)$ \\
    duality            & 1         & $\derived_\qc^-(X)$ & $\derived_\qc^+(Y)$ \\
                       & 2         & $\derived_\qc^-(X)$ & $\derived_\coh^+(Y)$ \\
                       & 3         & $\derived_\qc^-(X)$ & $\derived_\qc^\bounded(Y)$ \\
                       & 4         & $\derived(\Qcoh/X)$ & $\derived^+(\Qcoh/Y)$ \\
    \bottomrule
  \end{tabular}
  \caption{Overview of the configuration of the derived categories for each of the three parts of a Grothendieck duality context}
  \label{table:situations-derived-categories}
\end{table}

\subsection{How to recover classical results?}
\label{subsection:classical-results}


\section{History of the results}
\label{section:history}


\section{Applications}
\label{section:applications-grothendieck-duality}
\subsection{Local duality}
\label{subsection:local-duality}

\subsection{The yoga of six functors}
\label{subsection:six-functors-yoga}

\subsection{Fourier--Mukai transforms}
\label{subsection:fourier-mukai-transforms}

\printbibliography

\end{document}
